\section{Math extensions}

The math exetensions were inspired as I was making my way through
\cite{concrete-mathematics}, and various courses at DIKU. It seems to me that
these macros are generically useful.

\subsection{Groups}

Very often, mathematical expressions make use of grouping constructs such as
$\ceil{}$, $\floor{}$, $\parens{}$, etc. These constructs are relatively easy
to use in \LaTeX (with the \mono{amsmath} package), despite the fact that one
has to often distinguish between the left and right connectives, as with e.g.
\command{lfoor} and \command{rfloor}. What makes these groups particularly
impractical however, is that the height of the connectives is not automatically
adjusted to the content they enclose. To this end, one may resort to using the
commands \command{left} and \command{right}, as respective connective
prefixes\dots Yuk!  This lead to the specification of the following macros:

\begin{description}[\setleftmargin{80pt}\setlabelstyle{\commandstyle}]

\item [ceil\{group\}] $\ceil{\textt{group}}$

\item [floor\{group\}] $\floor{\textt{group}}$

\item [set\{group\}] $\set{\textt{group}}$

\item [seq\{group\}] $\seq{\textt{group}}$\quad (as in, \emph{sequence})

\item [card\{group\}] $\card{\textt{group}}$\quad (as in, \emph{cardinality})

\item [tuple\{group\}] $\tuple{\textt{group}}$

\item [parens\{group\}] $\parens{\textt{group}}$

\item [st\{group\}] $\st{\textt{group}}$\quad (as in, \emph{such that})

\end{description}


\subsection{Backus-Naur Form}

\begin{description}[\setleftmargin{90pt}\setlabelstyle{\commandstyle}]

\item [textt\{group\}] typewriter text in math mode.

\item [nonterm\{group\}] $\nonterm{\textt{group}}$

\item [term\{group\}] $\term{\textt{group}}$

\end{description}

\subsection{Cormen}

\begin{codebox}
\Procname{$\proc{Merge-Sort}(A,p,r)$}
\li \If $p < r$ \Then
\li $q \gets \floor{(p + r) / 2}$
\li $\proc{Merge-Sort}(A, p, q)$
\li $\proc{Merge-Sort}(A, q + 1, r)$
\li $\proc{Merge}(A, p, q, r)$
\End
\end{codebox}
